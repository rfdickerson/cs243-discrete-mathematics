%---------change this every homework
\def\yourname{Ronald Weasley}
% -----------------------------------------------------
\def\homework{0} % 0 for solution, 1 for problem-set only
\def\duedate{tues mar 31, 2014}
\def\duelocation{}
\def\hnumber{5}
\def\prof{Prof. Dickerson}
\def\course{cs243 - discrete - s'14}%-------------------------------------
%-------------------------------------

\documentclass[10pt]{article}
\usepackage[colorlinks,urlcolor=blue]{hyperref}
\usepackage[osf]{mathpazo}
\usepackage{amsmath,amsfonts,graphicx}
\usepackage{latexsym}
\usepackage[top=1in,bottom=1.4in,left=1.5in,right=1.5in,centering]{geometry}
\usepackage{color}
\definecolor{mdb}{rgb}{0.3,0.02,0.02} 
\definecolor{cit}{rgb}{0.05,0.2,0.45} 
\pagestyle{myheadings}
\markboth{\yourname}{\yourname}
\usepackage{clrscode}


\newenvironment{proof}{\par\noindent{\it Proof.}\hspace*{1em}}{$\Box$\bigskip}
\newcommand{\qed}{$\Box$}
\newcommand{\alg}[1]{\mathsf{#1}}
\newcommand{\handout}{
   \renewcommand{\thepage}{H\hnumber-\arabic{page}}
   \noindent
   \begin{center}
      \vbox{
    \hbox to \columnwidth {\sc{\course} --- \prof \hfill}
    \vspace{-2mm}
    \hbox to \columnwidth {\sc due \MakeLowercase{\duedate} \duelocation\hfill {\Huge\color{mdb}H\hnumber.}}
	\vspace{15pt}
	{\Huge\yourname}
      }
   \end{center}
   \vspace*{2mm}
}
\newcommand{\solution}[1]{\medskip\noindent\textbf{Solution:}#1}
\newcommand{\bit}[1]{\{0,1\}^{ #1 }}
%\dontprintsemicolon
%\linesnumbered
\newtheorem{problem}{\sc\color{cit}problem}
\newtheorem{practice}{\sc\color{cit}practice}
\newtheorem{lemma}{Lemma}
\newtheorem{definition}{Definition}

\begin{document}
\thispagestyle{empty}
\handout

\begin{enumerate}

\item (10 points) Arrange the functions 
$$3^n, 2^n, n 2^n,  n^{30}, (\log n)^3, \sqrt{n}\log^2 n, n\log n, \sqrt{n!}, n^{29}+n^{28}, n^{\sqrt{n}}$$
into increasing order of growth rates.

\item (10 points) 
To solve a particular problem you have access to two algorithms. 
The execution time of the first algorithm can be given as a function of the
input size $n$ as $f(n) = n^{1.5} \log^2 n$.
The execution time of the second algorithm is similarly: $g(n) = n^2$.
Which algorithm is faster asymptotically?
Is this algorithm faster for small $n$?
Find the minimum problem size $n$ needed so that the fastest asymptotic 
algorithm becomes faster than the other one.
Hint: limit your search in powers of 2. You may use calculators to help you
but the answer must self contained.

\item (10 points) 
What is the largest problem size $n$ that we can solve in no more than 
{\bf  one hour} using an algorithm that requires $f(n)$ operations, 
where each operation takes $10^{-9}$ seconds (this is close to a today's computer), 
with the following $f(n)$?
\begin{enumerate}
\item $\log_2 n$
\item $\log^5_2 n$
\item $4n$
\item $2n\log_2 n$
\item $\frac{n}{2}\log^2_2 n$
\item $n^2$
\item $(n/2)^3$
\item $2^n$
\item $n!$
\item $n^n$
\end{enumerate}

\item (10 points) Use pseudocode to describe an algorithm that determines whether a given function from a finite set to another finite set is one-to-one.

Hint: You may assume that the domain is $A=\{a_1,\ldots,a_m\}$ and the co-domain is $B=\{b_1,\ldots,b_n\}$. The function $f: A\rightarrow B$ is given as a set of pairs $\{(a_i,f(a_i))|\forall a_i\in A\}$.

\end{enumerate}


\end{document}