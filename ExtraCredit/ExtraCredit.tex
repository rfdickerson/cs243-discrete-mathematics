%---------change this every homework
\def\yourname{Melvil Dewey}
% -----------------------------------------------------
\def\homework{0} % 0 for solution, 1 for problem-set only
\def\duedate{tues april 21, 2014}
\def\duelocation{}
\def\hnumber{Extra}
\def\prof{Prof. Dickerson}
\def\course{cs243 - discrete - s'14}%-------------------------------------
%-------------------------------------

\documentclass[10pt]{article}
\usepackage{graphicx}
\usepackage[colorlinks,urlcolor=blue]{hyperref}
\usepackage[osf]{mathpazo}
\usepackage{amsmath,amsfonts,graphicx}
\usepackage{latexsym}
\usepackage[top=1in,bottom=1.4in,left=1.5in,right=1.5in,centering]{geometry}
\usepackage{color}
\definecolor{mdb}{rgb}{0.3,0.02,0.02} 
\definecolor{cit}{rgb}{0.05,0.2,0.45} 
\pagestyle{myheadings}
\markboth{\yourname}{\yourname}
\usepackage{clrscode}


\newenvironment{proof}{\par\noindent{\it Proof.}\hspace*{1em}}{$\Box$\bigskip}
\newcommand{\qed}{$\Box$}
\newcommand{\alg}[1]{\mathsf{#1}}
\newcommand{\handout}{
   \renewcommand{\thepage}{H\hnumber-\arabic{page}}
   \noindent
   \begin{center}
      \vbox{
    \hbox to \columnwidth {\sc{\course} --- \prof \hfill}
    \vspace{-2mm}
    \hbox to \columnwidth {\sc due \MakeLowercase{\duedate} \duelocation\hfill {\Huge\color{mdb}\hnumber.}}
	\vspace{15pt}
	{\Huge\yourname}
      }
   \end{center}
   \vspace*{2mm}
}
\newcommand{\solution}[1]{\medskip\noindent\textbf{Solution:}#1}
\newcommand{\bit}[1]{\{0,1\}^{ #1 }}
%\dontprintsemicolon
%\linesnumbered
\newtheorem{problem}{\sc\color{cit}problem}
\newtheorem{practice}{\sc\color{cit}practice}
\newtheorem{lemma}{Lemma}
\newtheorem{definition}{Definition}

\begin{document}
\thispagestyle{empty}
\handout

The objective for this extra credit opporunity is to target certain topic areas that you might need additional practice with. For your midterm exam, choose no more than 2 of the problems that you were weakest on. For each problem that you chose, lookup from the following correspondence list and do the \textbf{even} problems. For instance, if you scored low on Problem 4 on the midterm, you can answer the even questions 10--40  in the Rosen book from Chapter 1.4. You have the potential of adding 20 points on your midterm, capped to a maximum of 100\%. Please typeset, print, and submit your answer sheet on the due date to the instructor.

\begin{enumerate}
\item Problem 1: Propositional Logic - Chapter 1.1 problems 10--40
\item Problem 2: Digital Logic - Chapter 1.2 

\includegraphics{calcdisplay.eps}

Create a digital circuit that uses only NOT, AND, XOR, and OR gates to take as input a 4 bit number $a_3$,  $a_2$,   $a_1$,   $a_0$ and return a calculator-like display by lighting up the A, B, C, D, E, F, G lights on the display to represent that single digit.



\item Problem 3: Propositional Equivalences - Chapter 1.3 problems  10--40
\item Problem 4: Predicates and Quantifiers - Chapter 1.4 problems  10--40
\item Problem 5: Rules of Inference - Chapter 1.6 problems  10--40
\item Problem 6: Proofs - Chapter 1.7 problems  1--20
\item Problem 7: Proofs - Chapter 1.7 problems  22--42
\item Problem 8: Sets - Chapter 2.1 problems  10--40
\item Problem 9: Set Operations - Chapter 2.2 problems  10--40
\item Problem 10: Functions - Chapter 2.3 problems  10--40
\item Problem 11: Sequences and Summations - Chapter 2.4 problems  10--40
\item Problem 12: Puzzle - Chapter 1.2 20--36
\end{enumerate}


\end{document}